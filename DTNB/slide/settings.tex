%================================================================================
%usepackage
%--------------------------------------------------------------------------------
\usepackage{verbatimbox,bm,ulem,ifthen,wrapfig,colortbl,multicol,amsmath,color,
			listings,siunitx,tikz,pgfplots,caption}
\usepackage{graphicx}% for keepspectratio
%================================================================================
%Tikz setting
%--------------------------------------------------------------------------------
\usetikzlibrary{fit,calc,shapes,shapes.callouts,positioning,datavisualization,
				patterns,shapes.arrows}
%================================================================================
%PGF setting
%--------------------------------------------------------------------------------
\pgfplotsset{compat=newest,compat/show suggested version=false}
\usepgfplotslibrary{units}%need for color mapping
\usepgfplotslibrary{fillbetween}%need for NIDP plotting
%================================================================================
%Japanese setting and font setting
%--------------------------------------------------------------------------------
\usepackage{zxjatype}
%\usepackage{mathspec}
\usepackage[hiragino-pron]{zxjafont}
%\setallmainfonts{Fira Sans Light}
%================================================================================
%math rayout setting
%--------------------------------------------------------------------------------
% 数式(演算子など)のスペースを詰める
% =,→ 間の余白
\thickmuskip=1.0\thickmuskip
% +,- 間の余白
\medmuskip=0.8\medmuskip
% … などの装飾記号の余白
\thinmuskip=0.8\thinmuskip
% 行列を詰める
\arraycolsep=0.3\arraycolsep
% 数式の上下のスペースの変更
\AtBeginDocument{
  \abovedisplayskip     =0.5\abovedisplayskip
  \abovedisplayshortskip=0.5\abovedisplayshortskip
  \belowdisplayskip     =0.5\belowdisplayskip
  \belowdisplayshortskip=0.5\belowdisplayshortskip}
\setlength{\abovedisplayskip}{0pt}
\setlength{\belowdisplayskip}{0pt}
%================================================================================
%rayout setting
%--------------------------------------------------------------------------------
\renewcommand{\figurename}{Fig}
\setbeamersize{text margin left=5mm,text margin right=5mm} 
\usetheme[progressbar=frametitle,numbering=none]{metropolis} 
\setbeamercolor{frametitle}{bg=brown!20,fg=brown}
\makeatletter
%\setlength{\metropolis@frametitle@padding}{1.6ex}%frametitle height default
\setlength{\metropolis@frametitle@padding}{1.0ex}%frametitle height default
%\setlength{\metropolis@frametitle@padding}{0.8ex}%frametitle height default
\makeatother
%改行時行間ゼロ
\newcommand*{\narrowLS}{%行間ゼロ
  \setlength{\baselineskip}{1mm}}
%================================================================================
%color setting
%--------------------------------------------------------------------------------
\definecolor{cyan}{rgb}{0.0, 1.0, 1.0}
\definecolor{cottoncandy}{rgb}{1.0, 0.74, 0.85}
\definecolor{darkseagreen}{rgb}{0.56, 0.74, 0.56}
\definecolor{coral}{rgb}{1,0.5,0.31}
\definecolor{bleudefrance}{rgb}{0.19, 0.55, 0.91}
%================================================================================
%BOX setting
%--------------------------------------------------------------------------------
%title
\setbeamercolor{block title}{fg=white,bg=brown}
\setbeamercolor{block title alerted}{fg=white,bg=purple!50}
\setbeamercolor{block title example}{fg=white,bg=green!50!black}
%body
\setbeamercolor{block body}{fg=black!90,bg=brown!20}
\setbeamercolor{block body alerted}{fg=black,bg=purple!20}
\setbeamercolor{block body example}{fg=black!90,bg=green!20}
%================================================================================
%math setting
%--------------------------------------------------------------------------------
\newcommand{\kfrac}{\displaystyle\frac}
\newcommand{\ksum}{\displaystyle\sum}
\newcommand{\kint}{\displaystyle\int}
\newcommand{\klim}{\displaystyle\lim}
%================================================================================
%Setting the node of rectangle mainly for RMSF plot
%--------------------------------------------------------------------------------
\makeatletter
\tikzset{
    block filldraw/.style={% only the fill and draw styles
        draw, fill=yellow!20},
    block rect/.style={% fill, draw + rectangle (without measurements)
        block filldraw, rectangle},
    block/.style={% fill, draw, rectangle + minimum measurements
        block rect, minimum height=0.8cm, minimum width=6em},
    from/.style args={#1 to #2}{% without transformations
        above right={0cm of #1},% needs positioning library
        /utils/exec=\pgfpointdiff
            {\tikz@scan@one@point\pgfutil@firstofone(#1)\relax}
            {\tikz@scan@one@point\pgfutil@firstofone(#2)\relax},
        minimum width/.expanded=\the\pgf@x,
        minimum height/.expanded=\the\pgf@y}}
\makeatother
%================================================================================
%Define the frexible hatch
%--------------------------------------------------------------------------------
\makeatletter
       \pgfdeclarepatternformonly[\hatchdistance,\hatchthickness]{flexible hatch}
       {\pgfqpoint{0pt}{0pt}}
       {\pgfqpoint{\hatchdistance}{\hatchdistance}}
       {\pgfpoint{\hatchdistance-1pt}{\hatchdistance-1pt}}%
       {
           \pgfsetcolor{\tikz@pattern@color}
           \pgfsetlinewidth{\hatchthickness}
           \pgfpathmoveto{\pgfqpoint{0pt}{0pt}}
           \pgfpathlineto{\pgfqpoint{\hatchdistance}{\hatchdistance}}
           \pgfusepath{stroke}
       }
\makeatother
%================================================================================
